\documentclass{upmassignment}
\usepackage[spanish]{babel}
\usepackage{ifthen}
\usepackage{amsmath}
\usepackage{amsfonts}
\usepackage{booktabs}
\usepackage[x11names]{xcolor}
\usepackage{tcolorbox}
\usepackage{cclicenses}
\usepackage{url}

\usepackage{listings}
\lstset{basicstyle=\ttfamily,
  showstringspaces=false,
  commentstyle=\color{red},
  keywordstyle=\color{blue},
  backgroundcolor=\color{gray!30},
}


\usetikzlibrary{calc}



% Para mostrar/ocultar soluciones
\newboolean{show}
\setboolean{show}{true}
\setboolean{show}{false}
\usepackage{environ}
\NewEnviron{solucion}{
  \ifshow
      \begin{answer}\BODY\end{answer}
  \fi}






\coursetitle{Redes y Servicios}
\courselabel{RSER}
\exercisesheet{Broker MQTT y Grafana}{}
\student{\ }%
\semester{Primer Semestre 2024/2025}
\date{\today}
\university{Universidad Politécnica de Madrid}
\school{Departamento de Ingeniería de Sistemas Telemáticos}
%\usepackage[pdftex]{graphicx}
%\usepackage{subfigure}


\setlength{\textwidth}{5.0in}
\linespread{1.3}
\renewcommand{\PB}{{\bfseries Problema}}



\begin{document}



\section*{Introducción}

En esta práctica\footnote{Todas las preguntas
tienen el mismo valor/puntuación,
a excepción del Problema~8, que tiene
doble valor/puntuación.
}
vamos usar un broker
MQTT donde publicar las constantes vitales
de nuestro cliente. También utilizaremos
un datasource que se suscribirá al topic
donde publicamos mensajes, para poder
dibujar los datos en un dashboard de Grafana.

La siguiente figura resalta 
las componentes que trabajaremos:
\begin{figure}[h]
    \centering
\begin{tikzpicture}
    \node[draw,rectangle] (sensor) at (0,0)
        {sensor};
    \node[draw,rectangle] (script)
        at ($(sensor)+(0,-1)$)
        {\texttt{reporter.sh}};
    \node[draw,rectangle,anchor=west]
        (report)
        at ($(script.east)+(1,0)$)
        {\texttt{report.csv}};
    \node[draw,rectangle,fill=gray!30]
        (utils)
        at ($(report)+(0,1)$)
        {\texttt{utils.py}};
    \node[draw,rectangle,fill=gray!30]
        (client)
        at ($(utils)+(0,1)$)
        {\texttt{client.py}};
    \node[draw,rectangle,fill=gray!30,
        align=center]
        (server)
        at ($(client)+(2.5,0)$)
        {broker\\emqx};
    \node[draw,rectangle,fill=gray!30]
        (datasource)
        at ($(server)+(2.5,0)$)
        {datasource};
    \node[draw,rectangle,fill=gray!30]
        (grafana)
        at ($(datasource)-(0,1)$)
        {grafana};

    \draw[arrows=<->] (sensor.south)
        -- (script.north);
    \draw[arrows=->] (script.east)
        -- (report.west);
    \draw[arrows=<->] (report.north)
        -- (utils.south);
    \draw[arrows=<->] (utils.north)
        -- (client.south);
    \draw[arrows=<->] (client.east)
        --
        node[midway,above]
        {pub}
        (server.west);
    \draw[arrows=<->] (server.east)
        --
        node[midway,above]
        {sub}
        (datasource.west);
    \draw[arrows=<->] (datasource.south)
        --
        (grafana.north);



\end{tikzpicture}
\caption{escenario de la práctica.}
\label{fig:escenario}
\end{figure}


\section*{Instalación}

\subsection*{Instalación del broker EMQX}
Primero vamos a instalar el contenedor
del \emph{broker} EMQX ejecutando lo
siguiente en la máquina virtual:
\begin{lstlisting}[language=bash]
$ docker run -d --name emqx -p 1883:1883 -p 8083:8083\
  -p 8883:8883 -p 8084:8084 -p 18083:18083\
  emqx/emqx:5.0.2
\end{lstlisting}
El comando se encargará de descargar la
versión 5.0.2 de EMQX y crear un contenedor
llamado ``emqx''. Además, los puertos
locales indicados con las banderas \texttt{-p}
serán redireccionados al contenedor. Esto
último nos permitirá acceder al dashboard
de EMQX.
Además, con el comando de arriba
hacemos que el broker EMQX
\emph{haga disponible
la suscripción a topics} en el
puerto 1883.

Para acceder al dashboard del broker
EMQX acceda a la
siguiente URL en el navegador:
\begin{center}
\url{http://localhost:18083}
\end{center}
y haga login usando el usuario
``admin'' y la contraseña ``public''.
El dashboard mostrará paneles dedicados
a las conexiones MQTT abiertas con el
broker EMQX, así
como los topics, suscripciones, etc.


\subsection*{Instalación de Grafana}
Con Grafana vamos a poder visualizar las
métricas reportadas en cada uno de los
topics MQTT. En particular, vamos a
instalar grafana y un datasource de
MQTT. Un data source de grafana
es un plugin que permite recoger información
desde una fuente de datos. En nuestro caso,
la fuente de datos serán los distintos
topics MQTT.

Para instalar Grafana con el plugin
del datasource MQTT ejecute el siguiente
comando en la terminal:
\begin{lstlisting}[language=bash]
$ sudo docker run -d -p 3000:3000 --name=grafana \
  -e "GF_INSTALL_PLUGINS=grafana-mqtt-datasource" \
  grafana/grafana-enterprise
\end{lstlisting}
El comando descargará la imagen correspondiente
de Grafana y ejecutará el contenedor.

Para acceder al dashboard de Grafana
acceda a la siguiente URL en el navegador:
\begin{center}
\url{localhost:3000}
\end{center}
y haga login usando el usuario
``admin'' y contraseña ``admin''.



\section*{Publicación al broker EMQX}
A continuación vamos a ejecutar nuestro
cliente MQTT para publicar mensajes al
topic\footnote{Sustituya \texttt{\_\_Nombre\_\_}
por su nombre.}
\texttt{Madrid/Moncloa/\_\_Nombre\_\_}.
Para ello sírvase del programa
\texttt{client.py} de la anterior sesión de
laboratorio. 


Ejecute de nuevo el \texttt{reporter.sh}
y modifique el código del \texttt{client.py}
para publicar las constantes vitales
al broker EMQX instalado
en esta práctica.
A continuación haga los siguientes pasos:
\begin{enumerate}
    \item Inicie una captura wireshark
        en la interfaz \texttt{Loopback: lo};
    \item ejecute el \texttt{cliente.py}
        para observar la conexión y
        envío de, al menos, un mensaje; y
    \item detenga la captura wireshark y
        guárdela en el archivo\\
        \texttt{publish-broker-grupoX.pcapng}.
\end{enumerate}

\begin{problemlist}
    \pbitem Especifique el puerto
        destino TCP del paquete
        \texttt{CONNACK} (el acknowledgement
        de la conexión).
        Escriba la respuesta en el
        campo \texttt{ack\_tcp\_dst\_port}
        del archivo \texttt{respuestas-X.json}.
\end{problemlist}


El/los mensajes MQTT enviados por el cliente
podrá visualizarlos en el dashboard del
broker EMQX. Para ello basta con que acceda
a la URL:
\begin{center}
    \texttt{http://localhost:18083}
\end{center}
y fíjese en el panel
``Incoming messages''.






\section*{Fuente de datos para Grafana}
El siguiente paso es conectar Grafana
con el broker EMQX para poder visualizar
las constantes vitales.
Para ello es necesario crear una fuente
de datos de Grafana que se suscriba al topic
\texttt{Madrid/Moncloa/\_\_Nombre\_\_}
en que publica nuestro cliente.

Primero de todo, hay que buscar cuál
es la dirección IP del contenedor en el que
se ejecuta el broker EMQX. Para ello
ejecute el siguiente comando en su terminal:
\begin{lstlisting}[language=bash]
$ docker inspect `docker ps | grep 5.0.2\
  | cut -d' ' -f1` | grep "IPAddress"
\end{lstlisting}

\begin{problemlist}
\setcounter{enumi}{1}
\pbitem Anote la dirección IP del contenedor
    del broker EMQX en el campo
    \texttt{broker\_ip} de
    \texttt{respuestas-X.json}.
\end{problemlist}


Una vez hemos identificado la IP
del broker EMQX, vamos a proceder a 
crear la fuente de datos para Grafana.
Para ello siga los siguientes pasos:
\begin{enumerate}
    \item Acceda al panel de
        Grafana en la URL
        \url{localhost:3000};
    \item en el menú de la izquierda pinche
        en \texttt{Connections/Data Sources};
    \item pinche el botón
        \texttt{Add Data Source};
    \item pinche en \texttt{MQTT};
    \item rellene el campo \texttt{URI}
        con la IP del broker EMQX y
        el puerto en el que hace
        disponible la suscripción
        \texttt{IP:port} ;
    \item rellene los campos
        \texttt{username} y
        \texttt{password} usando los que
        utiliza en su script
        \texttt{client.py};
    \item inicie una captura wireshark
        en la interfaz \texttt{docker0};
    \item vuelva al dashboard de Grafana
        y pinche en el botón
        \texttt{Save \& test}; y
    \item detenga la captura de wireshark
        y guárdela -- aplicando el
        filtro ``mqtt' -- en el
        fichero
        \texttt{connect-data-source-grupoX.pcapng}.
\end{enumerate}
Tras seguir los pasos se habrá iniciado
una conexión MQTT entre la fuente
de datos de Grafana y el broker EMQX.


Sirviéndose de la captura de wireshark,
responda a las siguientes preguntas:
\begin{problemlist}
\setcounter{enumi}{2}
    \pbitem ¿Cuál es la QoS de la conexión
        entre el broker y la fuente de
        datos de Grafana?
        Anote su respuesta en el campo
        \texttt{qos\_grafana}
        de \texttt{respuestas-X.json}.
    \pbitem ¿Cuál es el ID que utiliza
        el plugin de la fuente de datos 
        de Grafana en la conexión MQTT?
        Anote su respuesta en el campo
        \texttt{id\_grafana}
        de \texttt{respuestas-X.json}.
\end{problemlist}






\section*{Creación de Dashboard de Grafana}
Ahora vamos a crear un dashboard de Grafana
para visualizar las constantes vitales
que reporta nuestro cliente MQTT.
Para ello haremos que nuestro plugin
del datasource MQTT se suscriba al topic
\texttt{Madrid/Moncloa/\_\_Nombre\_\_}.


Siga los siguientes pasos para crear el
dashboard:
\begin{enumerate}
    \item Pinche en \texttt{Dashboards} en el
        panel izquierdo de Grafana;
    \item pinche en el botón
        \texttt{Create Dashboard} y luego
        en \texttt{Add Visualization};
    \item pinche en
        \texttt{grafana-mqtt-datasource};
    \item inicie una captura wireshark
        en la interfaz \texttt{docker0};
    \item rellene el campo \texttt{topic}
        usando el topic en el que su cliente
        publica las constantes vitales;
    \item pinche con el ratón fuera del
        campo \texttt{topic} para que comience
        la visualización; y
    \item detenga la captura de wireshark
        y guárdela -- aplicando el
        filtro ``mqtt' -- en el
        fichero
        \texttt{data-source-subscribe-grupoX.pcapng}.
\end{enumerate}


Sirviéndose de la visualización del
dashboard y la captura de wireshark,
responda a las siguientes preguntas:
\begin{problemlist}
\setcounter{enumi}{4}
    \pbitem ¿Qué columna del dataset
        se está dibujando?
        Anote su respuesta en el campo
        \texttt{columna\_dibujada}
        de \texttt{respuestas-X.json}.

    \pbitem Identifique los mensajes
        de suscripción al topic que
        realiza el plugin del datasource
        MQTT.
        ¿Cuáles son los tipos de mensaje
        para el REQUEST y REPLY?
        Anote su respuesta en el campo
        \texttt{request\_ack\_types}
        de \texttt{respuestas-X.json}.
\end{problemlist}




\section*{Reporte de constantes en JSON}
A pesar de que su \texttt{client.py} está
enviando distintas constantes vitales --
cada una de ellas separada por comas --,
Grafana solo muestra un único valor
en el dashboard. Para mostrar las distintas
constantes vitales en el dashboard de
Grafana vamos a enviar las constantes
vitales en formato JSON.

\subsection*{Transformar línea en JSON}
Usaremos nuestro archivo \texttt{utils.py}
para crear una función
\texttt{line\_to\_json(line)} 
que tranforme una \texttt{line}
del dataset:
\begin{lstlisting}[language=bash]
462,462,97,18,100,36,Abnormal
\end{lstlisting}
en la siguiente cadena JSON
\begin{lstlisting}[language=bash]
'{"Idx":"462","Time":"462","HR":"97","RESP":"18",
"SpO2":"100","TEMP":"36","OUTPUT":"Abnormal"}'
\end{lstlisting}
Para ello sírvase de la librería
\texttt{json} y de la función
\texttt{json.dumps(obj)}, donde
\texttt{obj} es un diccionario Python.


\subsection*{Publicar el JSON}
Una vez programada la función
\texttt{line\_to\_json(line)},
actualice el \texttt{client.py} para
publicar un JSON en lugar de una línea.
Asegúrese de importar la función
\texttt{line\_to\_json} en las primeras
líneas del código de su cliente.

A continuación realice los siguientes
pasos:
\begin{enumerate}
    \item inicie una captura wireshark
        en la interfaz \texttt{Loopback: lo};
    \item ejecute su \texttt{client.py};
    \item deje que se publiquen diez
        de mensajes; y
    \item detenga la captura wireshark y
        guárdela en el archivo\\
        \texttt{publish-json-grupoX.pcapng}.
\end{enumerate}
Sirviéndose de la captura de wireshark,
responda a las siguiente pregunta:
\begin{problemlist}
\setcounter{enumi}{6}
    \pbitem ¿Qué campo del mensaje
        MQTT contiene el JSON?
        Anote su respuesta en el campo
        \texttt{campo\_json}
        de \texttt{respuestas-X.json}.

    \pbitem ¿Cuál es la longitud
        del mensaje\footnote{Numeramos los
        PUBLISH empezando en cero:
        $0,1,2,\ldots$} MQTT
        en el PUBLISH
        $X\mod{10}$ y $X+5\mod{10}$
        de su captura?
        Anote su respuesta en los campos
        \texttt{pub\_x} y
        \texttt{pub\_xm5}
        de \texttt{respuestas-X.json}.
\end{problemlist}




\section*{Visualización de datos con Grafana}
Para concluir vamos ilustrar las series
temporales de las constantes vitales.
Utilizaremos el datasource que hemos
configurado para obtener los datos del JSON
que reporta nuestro \texttt{client.py}.

Para crear el dashboard, siga
los pasos del apartado
``Creación de Dashboard de Grafana'',
sin iniciar una captura wireshark.

Acto seguido, procederemos para acceder
y transformar las distintas constantes vitales
reportadas en el JSON. Para ello siga
los siguientes pasos:
\begin{enumerate}
    \item Acceda a la pestaña
        \texttt{Transform Data};
    \item pinche en el botón
        \texttt{Add Transformation};
    \item pinche en el cuadrado
        \texttt{Convert Field Type};
    \item pinche en el botón
        \texttt{+ Convert Field Type};
    \item en \texttt{Select Field}
        elija, por ejemplo, el campo
        \texttt{TEMP} y en el campo
        \texttt{Type} seleccione
        \texttt{Number}.
\end{enumerate}
Observará la serie temporal de la temperatura.
Puede guardar el panel pinchando en el botón
\texttt{Save} que aparece arriba a la derecha.

\begin{problemlist}
    \setcounter{enumi}{8}
    \pbitem Tras guardar el dashboard,
        acceda a \texttt{Dashboards} en el
        menu izquierdo, y seleccione visualizar
        la información de los últimos
        5\,\textrm{min} (Last 5 minutes).
        Haga una captura/foto a la
        pantalla mostrando la serie temporal y
        llame al fichero de la imagen
        \texttt{serie-temporal}.
\end{problemlist}





\section*{Entrega}
\noindent Se subirá a moodle un archivo
\texttt{brokerX.zip}
(con \texttt{X} el número de grupo)
que contenga:
\begin{enumerate}
    \item el cliente \texttt{client.py};
    \item el archivo \texttt{utils.py};
    \item el JSON de respuestas \texttt{respuestas-X.json};
    \item las trazas de Wireskark:\\
          \texttt{publish-broker-grupoX.pcapng}\\
        \texttt{connect-data-source-grupoX.pcapng}\\
        \texttt{data-source-subscribe-grupoX.pcapng}\\
        \texttt{publish-json-grupoX.pcapng}; y
    \item la captura de pantalla 
        \texttt{serie-temporal}.
\end{enumerate}

\begin{tcolorbox}
    \textbf{Atención I}: las capturas
    deben contener
    \emph{solamente} tráfico MQTT.\\
    \textbf{Atención II}: una entrega
    sin los archivos especificados,
    o con archivos sin formato especificado
    tendrá un 0 en los \textbf{Problemas}
    correspondientes.
\end{tcolorbox}




% \section*{Introducción}
% 
% \noindent
% En esta práctica\footnote{Todas las preguntas
% tienen el mismo valor/puntuación.
% Cada apartado del Problema~2 tiene el
% mismo valor que el resto de Problemas.}\footnote{Este material está protegido por la
% licencia
% CC BY-NC-SA 4.0.
% \byncsa
% }
% vamos a crear un cliente
% MQTT
% \texttt{client.py}
% que enviará constantes vitales
% de un Sensor.
% Para ello se proporciona el programa
% \texttt{reporter.sh}
% (disponible en Moodle)
% que va añadiendo líneas a un archivo
% \texttt{report.csv}
% que almacena las constantes vitales
% de un sensor.
% 
% 
% 
% 
% 
% 
% \begin{figure}[h]
%     \centering
% \begin{tikzpicture}
%     \node[draw,rectangle] (sensor) at (0,0)
%         {Sensor};
%     \node[draw,rectangle] (script)
%         at ($(sensor)+(0,-1)$)
%         {\texttt{reporter.sh}};
%     \node[draw,rectangle,anchor=west]
%         (report)
%         at ($(script.east)+(1,0)$)
%         {\texttt{report.csv}};
%     \node[draw,rectangle]
%         (utils)
%         at ($(report)+(0,1)$)
%         {\texttt{utils.py}};
%     \node[draw,rectangle,fill=gray!30]
%         (client)
%         at ($(utils)+(0,1)$)
%         {\texttt{client.py}};
%     \node[draw,rectangle]
%         (server)
%         at ($(client)+(5,0)$)
%         {server};
%     \node[align=center,anchor=north]
%         at (server.south)
%         {\texttt{broker.emqx.io}\\
%         \texttt{port:1883}};
% 
%     \draw[arrows=<->] (sensor.south)
%         -- (script.north);
%     \draw[arrows=->] (script.east)
%         -- (report.west);
%     \draw[arrows=<->] (report.north)
%         -- (utils.south);
%     \draw[arrows=<->] (utils.north)
%         -- (client.south);
%     \draw[arrows=<->] (client.east)
%         --
%         node[midway,above,align=center]
%         {MQTT\\connection}
%         (server.west);
% 
% 
% 
% \end{tikzpicture}
% \caption{Escenario de la práctica.}
% \label{fig:escenario}
% \end{figure}
% 
% 
% \section*{Instalación de Dependencias}
% \noindent
% Antes de comenzar la práctica debemos
% instalar las dependencias necesarias
% para python y MQTT. 
% Descargue el fichero
% \texttt{practica-cliente.zip}
% de Moodle y descomprímalo en
% su carpeta personal.
% Abra una terminal y ejecute
% las siguientes líneas
% \begin{lstlisting}[language=bash]
% $ cd ~/practica-cliente
% $ ./dependencies.sh
% \end{lstlisting}
% 
% 
% 
% 
% 
% \section*{Generación de Constantes Vitales}
% \noindent Para generar el archivo
% \texttt{report.csv} con las constantes
% vitales, ejecute el programa
% que reporta las medidas del sensor:
% \begin{lstlisting}[language=bash]
% $ cd ~/practica-cliente
% $ ./reporter.sh
% \end{lstlisting}
% Tras ejecutar las líneas de arriba,
% verá que se ha generado un archivo
% \texttt{report.csv} con las siguientes
% columnas (separadas por comas):
% \begin{lstlisting}
% Idx, Time (s), HR (BPM), RESP (BPM), SpO2 (%),TEMP (*C),OUTPUT
% 0,0,94,21,97,36.2,Normal
% 1,1,94,25,97,36.2,Normal
% 2,2,101,25,93,38,Abnormal
% 3,3,55,11,100,35,Abnormal
% \end{lstlisting}
% \emph{Nota}:
% el archivo \texttt{report.csv} se genera
% de cero para cada ejecución del
% el \texttt{reporter.sh}. 
% 
% 
% \section*{Lectura de Constantes Vitales}
% \noindent 
% A continuación va a programar la
% función \texttt{read\_report()} encargada
% de leer constantes vitales del reporte.
% Esta función se encuentra en el archivo
% \texttt{utils.py} y se encarga
% de leer el archivo de reporte
% \texttt{report.csv}. Por ejemplo,
% podemos empezar en la tercera línea
% del archivo y leer dos líneas usando
% \begin{lstlisting}[language=bash]
% $ python3
% >>> import utils
% >>> utils.read_report('report.csv',10,2)
% ['9,9,94,26,97,29,Normal\n', '10,10,94,26,97,42,Abnormal\n']
% \end{lstlisting}
% 
% Para más detalles sobre el funcionamiento
% de \texttt{read\_report()}, lea
% la descripción de la función en el
% archivo \texttt{utils.py}.
% Tendrá que utilizar las funciones
% \texttt{open()} y \texttt{read\_lines()}
% para programar la función.
% 
% 
% \begin{problemlist}
%     \pbitem Programe la función
%     \texttt{read\_report()} y
%     ejecútela usando como parámetros
%     \texttt{LAST\_LINE=$X$}
%     y
%     \texttt{num\_lines=$\lceil \tfrac{X}{2} \rceil$},
%     donde $X$ es el número de grupo
%     asignado por el profesor.
%     Copie el resultado en el
%     campo \texttt{lectura} del
%     archivo
%     \texttt{respuestas-X.json}
%     usando dobles comillas:
% \begin{lstlisting}
% {
%     "lectura": [ "9,9,94,26,97,29,Normal\n",
%       "10,10,94,26,97,42,Abnormal\n"]
% }
% \end{lstlisting}
% \end{problemlist}
% \emph{Nota}: deje corriendo
% \texttt{reporter.sh} para que genere
% suficientes reportes.
% 
% 
% 
% \section*{Cliente MQTT}
% \subsection*{Conexión}
% \noindent
% A continuación va a programar parte
% de la lógica del cliente MQTT.
% En concreto se va a programar la conexión
% del \texttt{client.py} con el
% server ilustrada en la
% Figura.~\ref{fig:escenario},
% donde se especifica la dirección y puerto
% del servidor.
% 
% Para la comunicación con el servidor
% mediante MQTT,
% el programa \texttt{client.py} utiliza
% la librería \texttt{PAHO} de python.
% Los pasos para abrir una conexión
% (consulte la plantilla del cliente
% \texttt{client.py})
% son los siguientes:
% \begin{enumerate}
%     \item Conectarse con el servidor
%         MQTT: función
%         \texttt{connect\_mqtt()}; e
%     \item invocar el bucle de publicación
%         de mensajes:
%         función \texttt{publish()}.
% \end{enumerate}
% 
% \begin{figure}[h]
%     \centering
%     \begin{tikzpicture}
%         \node[draw,rectangle,fill=gray!30]
%             (client) at (0,0)
%             {\texttt{client.py}};
%         \node[draw,rectangle] (server) at
%             ($(client)+(8,0)$)
%             {server};
%         \node[align=center,anchor=north]
%             at (server.south)
%             {\texttt{broker.emqx.io}\\
%             \texttt{port:1883}};
% 
%         \draw[arrows=<->]
%             (client.north east)
%             --
%             node[midway,above]
%             {\texttt{connect\_mqtt()}}
%             (server.north west);
%         \draw[arrows=->]
%             (client.south east)
%             --
%             node[midway,below]
%             {\texttt{publish()}}
%             (server.south west);
%     \end{tikzpicture}
%     \caption{Conexión y publicación del
%     cliente.}
%     \label{fig:connect-publish}
% \end{figure}
% 
% 
% Modifique la plantilla de
% \texttt{client.py} para conectarse
% al servidor y responda a las siguientes
% preguntas.
% 
% \begin{problemlist}
%     \stepcounter{enumi}
%     \pbitem Inicie una captura wireshark
%         en todas las interfaces
%         (\texttt{any}) poniendo filtro
%         ``mqtt''. Ejecute en la consola
%         \texttt{client.py} e identifique
%         las tramas de conexión y ACK
%         en wireshark. Detenga la captura
%         y rellene las
%         siguientes preguntas en el
%         JSON de respuestas:
%         \begin{enumerate}
%             \item ¿Cuál es el ID de su cliente?
%             \item En la captura wireshark,
%                 ¿cuál es el número de trama
%                 del ACK de la conexión?
%         \end{enumerate}
%         Guarde\footnote{Para guardas solo los paquetes MQTT
%                 basta con poner en el filtro
%                 ``mqtt'',
%                 pinchar en
%                 File-Export Specified Packets,
%                 y en el cuadro ``Packet
%                 Range'' pinchar en la
%                 bola  ``Displayed''.}
%                 la captura en el fichero
%         \texttt{captura-connect-grupoX.pcapng}.
% Siga el mismo procedimiento del
% pie de página para todas las capturas
% de la práctica.
% \end{problemlist}
% 
% 
% 
% \subsection*{Publicación}
% \noindent A continuación debe
% modificar el \texttt{client.py} para
% leer los datos del sensor
% (archivo \texttt{report.csv}) y publicarlos
% en el topic
% \texttt{rserGX/vitals}, donde X
% es su número de grupo.
% 
% La función \texttt{publish()}
% contiene un bucle que continuamente publica
% mensajes MQTT. Modifique el bucle
% para que espere
% $(X\mod{5}) + 5$ segundos al final de cada
% iteración.
% Además, modifique \texttt{publish()}
% para invocar a \texttt{read\_report()} y
% publicar \emph{una a una}
% las constantes vitales.
% %% En concreto,
% %% debe enviar el valor de
% %% la columna \texttt{Time (s)}
% %% (columna número 1)
% %% y la columna
% %% $(X \mod{4}) + 2$.
% 
% 
% 
% Tras haber realizado las modificaciones
% indicadas, inicie una captura en
% wireshark filtrando paquetes MQTT y
% responda a las siguientes
% preguntas usando la captura entregada.
% Para ello responda en
% en el JSON de respuestas a:
% 
% 
% \begin{problemlist}
%     \stepcounter{enumi}
%     \stepcounter{enumi}
%     \pbitem ¿En qué instante
%         se envía la muestra $X$?
%         (valor ``Time'' en wireshark).
% 
%     \pbitem ¿Cuánto tiempo
%         pasa entre dos 
%         Ping de MQTT?
% \end{problemlist}
% 
% Guarde la captura en el fichero
% \texttt{captura-publish-grupoX.pcapng}.
% 
% 
% \subsection*{Desconexión y QoS}
% \noindent A continuación vamos a probar
% los distintos niveles de QoS ofrecidos
% por MQTT. Para ello basta con
% añadir el argumento
% \texttt{qos=$q$} a la función
% \texttt{client.publish()}, donde
% $q\in\{0,1,2\}$ especifica el nivel
% QoS de MQTT.
% 
% En primer lugar vamos a probar
% qué sucede con un Qos de 0 si se
% cae la conexión. Para ello vamos a
% apagar la interfaz
% \texttt{eth0} usando el
% siguiente comando
% \begin{lstlisting}[language=bash]
% $ ip link set down dev eth0
% \end{lstlisting}
% y la encenderemos con el siguiente comando
% \begin{lstlisting}[language=bash]
% $ ip link set up dev eth0
% \end{lstlisting}
% 
% 
% 
% \subsection*{QoS 0}
% \noindent
% Ponga un QoS 0 y ejecute el
% \texttt{client.py} usando el
% siguiente comando para guardar la salida:
% \begin{lstlisting}[language=bash]
% $ python3 client.py 2>&1 | tee /tmp/qos0-grupoX.log
% \end{lstlisting}
% 
% Inicie una
% captura en wireshark filtrando
% tráfico MQTT y espere a que
% se envíen, al menos, dos tramas
% PUBLISH. Apague la interfaz
% \texttt{eth0} y espere a que
% el cliente imprima
% ``Failed to send message [...]''.
% Vuelva a encender la interfaz.
% 
% 
% Responda a las siguientes preguntas
% en el JSON de respuestas:
% \begin{problemlist}
%     \stepcounter{enumi}
%     \stepcounter{enumi}
%     \stepcounter{enumi}
%     \stepcounter{enumi}
%     \pbitem Especifique el \verb|Idx|
%         de las muestras identificadas
%         como perdidas por el cliente.
%     \pbitem Especifique el \verb|Idx|
%         de las muestras que no se
%         han enviado con éxito.
% \end{problemlist}
% Detenga la captura de wireshark y
% guárdela en el fichero
% \texttt{qos0-grupoX.pcapng}.
% 
% 
% \subsection*{QoS 1}
% \noindent
% Ponga un QoS 1 y ejecute el \texttt{client.py}
% como en el apartado anterior, esta vez
% guardando el log en el archivo
% \texttt{qos1-grupoX.log}.
% 
% De nuevo, inicie una captura en wireshark
% filtrando tráfico MQTT y espere a que
% se envíen, al menos, dos tramas PUBLISH.
% Apague la interfaz \texttt{eth0}
% y espere de nuevo a que el cliente
% imprima ``Failed to send message [...]''.
% Vuelva a encender la interfaz.
% 
% 
% 
% 
% 
% \begin{problemlist}
%     \setcounter{enumi}{6}
%     \pbitem Indique el
%         \texttt{Message Identifier}
%         de los mensajes duplicados.
% 
% 
%     \pbitem Repita otra captura
%         incrementando el \texttt{keepalive}
%         al doble e indique el
%         \texttt{Message Identifier}
%         de los mensajes perdidos.
% 
% \end{problemlist}
% Guarde ambas capturas
% en los ficheros
% \texttt{qos1-grupoX.pcapng} y\\
% \texttt{qos1-x2keepalive-grupoX.pcapng}.
% 
% 
% 
% 
% \section*{Entrega}
% \noindent Se subirá a moodle un archivo
% \texttt{clienteX.zip}
% (con \texttt{X} el número de grupo)
% que contenga:
% \begin{enumerate}
%     \item el cliente \texttt{client.py};
%     \item el archivo \texttt{utils.py};
%     \item el JSON de respuestas \texttt{respuestas-X.json};
%     \item las trazas de Wireskark
%         \texttt{captura-connect-grupoX.pcapng},\\
%         \texttt{captura-publish-grupoX.pcapng},
%         \texttt{qos0-grupoX.pcapng},\\
%         \texttt{qos1-grupoX.pcapng},\
%         \texttt{qos1-x2keepalive-grupoX.pcapng}; y
%     \item los logs
%         \texttt{qos0-grupoX.log},
%         \texttt{qos1-grupoX.log}.
% \end{enumerate}
% 
% \begin{tcolorbox}
%     \textbf{Atención I}: las capturas
%     deben contener
%     \emph{solamente} tráfico MQTT.\\
%     \textbf{Atención II}: una entrega
%     sin los archivos especificados,
%     o con archivos sin formato especificado
%     tendrá un 0 en los \textbf{Problemas}
%     correspondientes.
% \end{tcolorbox}


\end{document}
